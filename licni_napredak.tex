\documentclass[b5paper]{article}

\usepackage[serbianc]{babel}
\usepackage[b5paper, lmargin=0.1\paperwidth, rmargin=0.1\paperwidth, tmargin=0.1\paperheight, bmargin=0.1\paperheight]{geometry} %margins

\edef\restoreparindent{\parindent=\the\parindent\relax}
\usepackage{parskip}
\restoreparindent

\begin{document}

\large

Јеврејска пословица коју је Владета Јеротић често цитирао каже да "ако не постанемо бољи, постаћемо гори". Џордан Питерсон као средишњи мотив митологије свих народа види борбу између реда и нереда, односно познатог и непознатог. По његовом виђењу, спољашње околности приморавају појединца да се суочи са непознатим. Уколико успешно одговори на изазов, појединац излази јачи и способнији за живот. Уколико одбије суочавањем са непознатим, појединац бива кажњен.

Очигледно је да тема односа појединца према свету, као и класификација начина понашања који се сматрају одрживим, односно неодрживим није нова. Ипак, као нови феномен, почетком прошлог века, јавио се жанр књига грубо назван "лични напредак". Најутицајнија књига приликом популаризације овог жанра била је Како освојити пријатеље и утицати на људе Дејла Карнегија (How to Win Friends and Influence People, Dale Carnegie). Бестселер новијег датума је 7 навика успешних људи Стивена Ковија (The 7 Habits of Highly Effective People, Stephen R. Covey).

Поред две књиге које су најпознатији представници жанра, постоји безброј издања чији су главни мотиви: успех (апстрактан појам), продуктивност, партнерски односи. Као проблем, писци издвајају појединца и његове "неправилне" обрасце размишљања. Неправилни су они обрасци који не криве појединца за ситуацију у којој се налази и који не "сагледавају ширу слику", односно, који не воде рачуна "о интересу свих страна". Променом начина размишаљања у "позитиван" обећавају читаоцу излазак на пут који ће га учинити бољим од других. Битан мотив овог жанра је такође да "није битно победити у игри, већ бити играч".

Иако савети звуче здраворазумски, поставио бих следеће питање: шта ново они доносе у односу на класично васпитање и елементарне вештине понашања у друштву? Због тога, желео бих да анализирам околности које су довеле до популарности овог жанра. Неопходно је разумети околности у друштву (пре свега, индустријализацију) и њихов утицај на појединца. Такође, није занемарљив ни специфичан утицај који читање књига оставља на читаоце.

Као главни извор осећања издвојености и отуђености у појединцу, Ерих Фром види индустријализацију, живот у великим градовима као и нестанак традиционалног система вредности. Уско специјализована занимања као и велики градови препуни реклама и огромних зграда чине да се појединац осећа незнатно у односу на систем који је сам створио. Нестанак традиционалног система вредности, иако је ослободио човека традиционалних оквира, није понудио нови правац развоја. Излаз из овог осећаја безначајности је био "бекство од слободе" у тоталитарна друштва комунизма и фашизма. 

"Најбоља је она књига која ти говори оно што већ знаш", говорио је Џорџ Орвел. Речник и књижевни стил ових књига је крајње једноставан (односно, сиромашан). Може се слободно рећи да овај жанр представља карикатуру времена у којем живимо. Са једне стране притиснут нереалним очекивања, а са друге стране ограничен својим скромним могућностима, читалац верује да ће наћи спас. Читајући савете, види своју сврху у неоспорно хаотичном окружењу.

Иако је употребна вредност ових савета скромна (усудио бих се да кажем и безачајна), оне читаоца испуњавају оптимизмом. Стога, можемо рећи да оваква литература представља својеврсан плацебо. У непредвидивом окружењу, оне констатују улогу појединца и уверавају га да је на путу напретка, чак и у комплетном одсуству објективних мера попут повећања плате или рада на одговорнијим задацима.

Стога, да ли их можемо сматрати безазленом појавом? Рекао бих никако! Улепшавање истине и отворено лагање читалаца никако није безазлено! Бављење споредним уместо суштинским темама наноси штету напретку друштва! Објективни утицај ових књига је кочење истинског личног напретка. Који је онда прави начин за остваривање личног напретка? По мојем виђењу, лични напредак се састоји из три нераздвојне целине. Прву целину је најједноставније описати (али најтеже применити) - суочавање са свакодневим изазовима живота. Не постоје два иста живота, те није могуће ни да постоји објективна формула за успех. Другу целину чини изучавање стручне литературе из одређене струке, као и праћење релеватних дешавања и трендова у струци. Трећу целину чини изучавање уметности, која омогућава увид у времена, ситуације и животе можда нећемо доћи непосредно у контакт. На жалост, ово је вероватно најпотцењенији вид личног напретка.
\end{document}
