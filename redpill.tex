\documentclass[b5paper]{article}

\usepackage[serbianc]{babel}
\usepackage[b5paper, lmargin=0.1\paperwidth, rmargin=0.1\paperwidth, tmargin=0.1\paperheight, bmargin=0.1\paperheight]{geometry} %margins
\usepackage{float}
\usepackage{graphicx}
\usepackage{caption}
\usepackage{subcaption}
\usepackage{hyperref}

\edef\restoreparindent{\parindent=\the\parindent\relax}
\usepackage{parskip}
\restoreparindent

\newcommand{\websource}[1]{\caption*{Преузето са: \href{#1}{#1}}}

\begin{document}

\large

У претходном тексту, обрађен је феномен личног напретка. Тема овог текста је \textit{redpill} покрет, далеко радикланији и штетнији покрет који прати исту основну црту - превазилажење осећања мање вредности појединца. Покрет је свој назив добио по чувеној сцени из филма Матрикс. Главном лику Неу понуђен је избор између \textit{плаве пилуле} - живота у илузији и \textit{црвене пилуле} - живота у стварности.

Појмови сродни појму \textit{redpill} су: \textit{manosphere} - заједница покрета који се баве проблемима мушкараца, \textit{incel (invlountary celibate)} - мушкарци ускраћени женске пажње супротно њиховој вољи, \textit{pick-up artists (PUAs)} - инструктори завођења који своје ученике подучавају вештинама комуникације са женама, \textit{black pill} - фаталистички поглед на будућност света и положај мушкараца у њему. Како видим све ове покрете као варијацију на исту тему и као изражавање истог осећања, у даљем тексту користићу израз \textit{redpill}. Оно што ове покрете разликује је једино интензитет штете нанет појединцу, а који варира од блаже фрустрације и потиснутог самоповређивања (\textit{redpill}) до мржње и отвореног самоповређивања (\textit{incel}, \textit{black pill}).

За присталице овог покрета, живот у стварности представља признање незавидне позиције мушкарца у данашњем друштву. По њиховом виђењу, мушкарац који није натпросечно материјално обезбеђен је небитан, те је сасвим природно да није поштован у друштву и да није привлачан женама. Милијарде прегледа сведоче о распрострањености оваквог осећања. Али, велика је заблуда рећи да је оваква појава нова.

По виђењу Ериха Фрома, ндустријска револуција довела је до нестанка традиционалног друштвеног поретка. Порекло више није гарантовало статус у друштву, а нестабилно и непредвидиво тржиште више није гарантовало стабилност прихода. Индустријски процес производње који посао дели на мале целине и живот у великим градовима додатно су подстицали осећање издвојености појединца. Препознајући тако пољуљане основне психолошке потребе у друштву, на власт је дошао Адолф Хитлер.

Закључак Ериха Фрома је да би појава постала масовна, неопходно је да задовољава психолошке потребе већине чланова друштва. Осећање издвојености у друштву описано у књизи Бекство од слободе је и данас итекако присутно. Оно што наше доба разликује од почетка прошлог века је, по виђењу корејског филозофа Бјунг-Чул Хана (Byung-Chul Han) прелазак са \textit{друштва дисциплине} на \textit{друштво остварења}.

Како приморавање људи на већу продуктивност дошло до свог максимума, створен је наратив о неопходности личних остварења. Једно огледало утицаја овог наратива је покрет личног напретка. Оно што мету жртви \textit{redpill} покрета разликује је далеко веће осећање усамљености. Такво осећање, на жалост, сасвим је утемељено у стварности - број блиских пријатеља је све мањи.

Део који није утемељен у стварности је виђење комплетног непријатељства друштва према појединцу. Ово иде и до граница психолошког самоповређивања. На жалост, због одсуства видљивих ожиљака, појава је теже уочљива јавности, иако штета по појединца може да буде и већа у односу на физичко самоповређивање.

Чак ни Ендру Тејт, икона \textit{redpill} покрета, није нова појава. У књизи Познавање човека, Алфред Адлер је описао \textit{нападачки тип} личности. Они своју храброст уздижу до бахатости, а неопходно је и наглашавање успеха, како себи, тако и другима. Осећање страха превазилазе покушајем огуглавања, док нежност потискују као слабост. Кроз \textit{redpill} покрет се види ванвременост Адлерове мисли.

Прича Ендру Тејта се своди на низ сличних изјава, чија је главна парадигма лична борба против свеприсутних неповољних околности. По његовом виђењу, мушкарац се рађа небитан и једини његов задатак је да се у животу оствари стичући поштовање од стране других мушкараца, али и жена. Једини начин за ово је победа у борби са другим мушкарцима, која се пре свега огледа у већој способности, снази и количини зарађеног новца. Такође, негира постојање депресије и анксиозности.

Депресију види као заверу чији циљ је уништавање мушкости. Сматра да је депресија осећање, и да као таква не може да буде стварна. Види је као израз себичности, пошто депресивна особа размишља о својим осећањима, која њега не занимају. Сматра да је за пораст депресије у свету криво слављење слабости. Анксиозност види као нормалну реакцију на неспособност појединца пред животним изазовима, наглашавајући да он не може да буде анксиозан јер је успешан у свему.

Негирањем важности људских осећања, као и сталним наглашавањем неопходности борбе против слабости, очигледно је да Ендру Тејт покушава да савлада своје лично осећање мање вредности. Наглашавање неопходности преузимања контроле осликава уплашеност пред непредвидивим околностима. Наглашавање важности посвећености послу и пословним успесима поред превазилажења личног осећања мање вредности осликава и нездрав дух времена у којем живимо.

Иако издвојен као најутицајна личност ове сфере, Ендру Тејт није једини гласноговорник ове погубне идеологије. А и као сви негативни трендови, полако долази и код нас. Тако у скорије време Немања Антић, по струци економиста, освојио је пажњу домаће публике. У основи, порука је слична Тејтовој. Нешто већи акценат је стављен на неопходност надпросечног радног времена. Контрадикторно томе, сматра да су кућа и викендица нешто на чему треба радити цео живот. Шта даје смисао натпросечном раду, када су до сада људи са далеко скромнијим примањима то могли да приуште? Такође, сматра да се послом не треба бавити због новца, већ због љубави према послу. Сложио бих се да је погрешно бавити се послом који не волимо, али исто тако је бесмислено бавити се послом који није профитабилан. Кроз причу, такође се помиње одбијање од стране жена.

Не смемо да будемо заварани оваквим, наизглед аполитичним наративом. Кроз њега је вешто проткана (по речима Бјунг Чул Хана) \textit{психополитика неолиберализма}. Оно што ову пропаганду чини различитом је да уместо традиционалне пропаганде која је била заснована на разуму, је да је она заснована на \textit{емоцијама}. Тешко је одупрети се од удараца испод појаса, метафорички речено. Крајњи исход овог утицаја ће бити епидемија депресије. Револуције више неће бити могуће због уверења да смо \textit{господари своје судбине}, и да је кривица искључиво индивидуална.

Амерички психотерапеут Данијел Маклер (широј јавности познат по снимању документарних филомова о лечењу писхоза без употребе антипсихотика) увео је концепт \textit{фантазије о спасењу од стране родитеља} (parental rescue fantasy). По његовом виђењу, деца чије основне потребе у детињству нису задовољене од стране родитеља ову фантазију преносе у одрасло доба, очекујући да ће неко други да се брине за њих безусловно. Исто тако, задовољење ове фантазије је могуће пружањем помоћи, за коју се за узврат добија љубав и пажња.

Свеприсутност интернета учинила је да људи на све шири скуп питања одговор траже \textit{гуглањем}. Стога, није чудно очекивати да се одговори на животна питања на која се наилази на путу стицања зрелости такође траже на интернету, посебно од стране млађе популације која је склонија употреби технологије. Алгоритми друштвених мрежа су дизајнирани тако да нуде садржај сличан већ гледаном, чиме се додатно учвршћују виђења којима је корисник склон. Како се животне околности јесу нагло промениле, посматрање наратива различитог од погледа родитеља може да остави уверљив утисак.

Део публике је неоспорно незрео, што због својих година старости, или због одрастања под неповољним породичним и животним околностима. Други део је жртва преувеличавања и интернализовања захтева данашњице - бесконачног остварења, рада до изнемоглости. У оба случаја, штета је велика јер се у преломним животним тренуцима појединцу не нуди садржај који води стварном напретку и сазревању, већ садржај који води уверљивом привиду. Код млађих гледалаца, штета је далеко већа због мање развијене способности критичког размишљања.

За разлику од личног напретка, који појединца усмерава на бескорисне, али релативно безазлене навике (попут јутарње рутине медитације, писања дневника или свакодневног писања три циља), \textit{redpill} уверења наносе директну штету менталном здрављу. Отуђење појединаца у друштву, као и све већа економска раслојавања јесу озбиљна претња даљем напретку друштва. Сасвим је и природно да у појединцу изазивају нелагоду, како би био подстакнут на њихово превазилажење. Акција је боља очајања, зар не? Зашто би било спорно рећи да је борба као приступ животу једино одрживо решење? Нисмо ли и ми Срби, на крају крајева, потомци бораца који су својом жртвом у више наврата утицали на историју Европе?

На жалост, вид борбе промовисан кроз \textit{redpill} идеологију није исти, без обзира на бројне покушаје да се представи као угледање на славну прошлост, у односу на коју смо доживели морални и карактерни пад. Борба наших предака била је заснована и мотивисана позитивним идеалима - љубави према потомцима као и вери и култури. Жртва је била виђена као неопходан део сарадње ради колективног опстанка. Насупрот томе, \textit{redpill} констатује вечно непријатељство појединаца. Такво гледиште не чини да се човек на достојанствен начин помири са својим животним разочарењима - а што је неопходан предуслов за сазревање. Оваква борба не проистиче из љубави и вере уз будућност, већ из \textbf{неуротичног страха од будућности}. Такође, не подстиче га на заједништво, јер сви су ту да би били бољи од нас, узели сав капитал као и све жене овог света, зар не? А и сви чекају да престану да се друже са нама када нас престигну!

Уколико међу читаоцима има људи који су тренутно у фази угледања на \textit{redpill}, верујем да су последње две реченице изазвале реакцију од благе непријатности до пробадања у грудима. Разлог томе је да су тенденциозно написане да загребу у најболније делове психе, што је честа \textit{redpill} техника. Честа опаска упућена критичарима личног напретка и \textit{redpill}-а је да такав садржај није лош уколико некоме помогне. На жалост, као заговорник подизања свести о менталном здрављу не могу да се сложим са таквом констатацијом. Не смемо да одобравамо садржај из области менталног здравља уколико он не подстиче појединца на темељно и потпуно сазревање. Народски речено - не треба да дозволимо ћоравом да води слепог. Наравно, иако аутори себе не сматрају психотерапеутима, њихов садржај конзумирају искључиво људи који имају потешкоће са осећајем несигурности као и животним разочарењима, те му морамо судити по датим критеријумима.

Како живимо у добу у којем је један од главних наратива бесконачно остварење појединца, може деловати природно да је начин опстанка и успеха заправо терање ината само себи. Овде и лежи замка, јер човек није себи непријатељ, а његове природне потребе (попут одмора или хобија) нису рђаве! Баш напротив, оне су неопходан и нераскидив део личности. Тешко је оценити да ли кроз овај вид мазохизма разочарани појединци налазе врсту утехе. Највероватније да то виђење важи барем за један део активне публике. Психолошко самоповређивање и задовољство које доноси је феномен који је тежак за разумевање онима који га нису проживели, без обзира на отвореност духа.

У овом тренутку, тешко је рећи да ли је више присталица или противника овог покрета. Ипак, неоспорно је да је број противника у порасту. Разлог овоме је буђење појединаца и увиђање да су ишли погрешним путем. На жалост, многи су морали да плате сурову цену прегоревања (енгл. burnout) не би ли се освестили. Ипак, како се дух времена неће нагло променити, \textit{redpill} ће бити актуелан још дуго. Како је интернет доступан свима, биће активна замка у коју ће упадати делом наивни људи (који ће видети руку спаса у свету који не разумеју), а делом људи који су карактерно агресивнији (који ће видети огледало себе).

Осим дизања свести од стране струке (пре свега социолога и психолога), не видим друга решења. Како је у овој анализи показано да је распрострањеност утемељена у самој друштвеној клими, не можемо наћи једног кривца. Проблем који је проткан кроз само друштво је тешко сагледати, а још теже решити. Ово наравно није позив на очајавање. Још једном бих нагласио неопходност образовања, којим би требало да се баве морални појединци. Дебату у којој психотерапеут пажљиво саслушава аргументације Ендру Тејта и слаже се са њима не могу назавти другачије до \textbf{издаја струке}.

Како је све више психолога који су посвећени управо помагању људи у остваривању сличних циљева као и \textit{redpill}, бојим се да ће чак и тражење стручне помоћи бити све мање корисно. Изгледа да је запажање Данијела Маклера да су психотерапеути често мање здрави од клијената тачно. Појединцу којег прогања осећање издвојености и одбачености није могуће обратити се на начин који износи искључиво разумну аргументацију. Они су пре свега привучени изражавањем саосећајности (макар и кроз констатације о њиховом болу, иако су често све сем саосећајне).

Да би сузбили ову пошаст, морамо прво изразити разумевање и саосећање за ситуацију у којој се налази појединац. Тек са тако стеченим поверењем, можемо постепено усмерити појединца на пут истинског сазревања. А зрелост у данашњим временима итекако подразумева вештину ношења са небитношћу и неупадљивошћу. Без круга пријатеља, као и односа са делом заједнице, овај проблем није премостив. Али, уколико не стичемо поверење са искреном намером, занат нам је проклет. Психотерапеутски рад никада није био једноставан, те се морамо наоружати пре свега стрпљењем! Не можемо очекивати да се кроз краћу придику пређе пут од неурозе до сагледавања суштинског проблема друштва.

\end{document}
