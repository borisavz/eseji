\documentclass[b5paper]{article}

\usepackage[serbianc]{babel}
\usepackage[b5paper, lmargin=0.1\paperwidth, rmargin=0.1\paperwidth, tmargin=0.1\paperheight, bmargin=0.1\paperheight]{geometry} %margins
\usepackage{float}
\usepackage{graphicx}
\usepackage{caption}
\usepackage{subcaption}
\usepackage{hyperref}

\edef\restoreparindent{\parindent=\the\parindent\relax}
\usepackage{parskip}
\restoreparindent

\newcommand{\websource}[1]{\caption*{Преузето са: \href{#1}{#1}}}

\begin{document}

\large

У претходном тексту, обрађен је феномен личног напретка. Тема овог текста је \textit{redpill} покрет, далеко радикланији и штетнији покрет који прати исту основну црту - превазилажење осећања мање вредности појединца. Покрет је свој назив добио по чувеној сцени из филма Матрикс. Главном лику Неу понуђен је избор између \textit{плаве пилуле} - живота у илузији и \textit{црвене пилуле} - живота у стварности.

Појмови сродни појму \textit{redpill} су: \textit{manosphere} - заједница покрета који се баве проблемима мушкараца, \textit{incel (invlountary celibate)} - мушкарци ускраћени женске пажње супротно њиховој вољи, \textit{pick-up artists (PUAs)} - инструктори завођења који своје ученике подучавају вештинама комуникације са женама, \textit{black pill} - фаталистички поглед на будућност света и положај мушкараца у њему. Како видим све ове покрете као варијацију на исту тему и као изражавање истог осећања, у даљем тексту користићу израз \textit{redpill}. Оно што ове покрете разликује је једино интензитет штете нанет појединцу, а који варира од блаже фрустрације (\textit{redpill}) до отвореног самоповређивања (\textit{incel}, \textit{black pill}).

За присталице овог покрета, живот у стварности представља признање незавидне позиције мушкарца у данашњем друштву. По њиховом виђењу, мушкарац који није натпросечно материјално обезбеђен је небитан, те је сасвим природно да није поштован у друштву и да није привлачан женама. Милијарде прегледа сведоче о распрострањености оваквог осећања. Али, велика је заблуда рећи да је оваква појава нова.

По виђењу Ериха Фрома, ндустријска револуција довела је до нестанка традиционалног друштвеног поретка. Порекло више није гарантовало статус у друштву, а нестабилно и непредвидиво тржиште више није гарантовало стабилност прихода. Индустријски процес производње који посао дели на мале целине и живот у великим градовима додатно су подстицали осећање издвојености појединца. Препознајући тако пољуљане основне психолошке потребе у друштву, на власт је дошао Адолф Хитлер.

Закључак Ериха Фрома је да би појава постала масовна, неопходно је да задовољава психолошке потребе већине чланова друштва. Осећање издвојености у друштву описано у књизи Бекство од слободе је и данас итекако присутно. Оно што наше доба разликује од почетка прошлог века је, по виђењу корејског филозофа Бјунг-Чул Хана (Byung-Chul Han) прелазак са \textit{друштва дисциплине} на \textit{друштво остварења}.

Како приморавање људи на већу продуктивност дошло до свог максимума, створен је наратив о неопходности личних остварења. Једно огледало утицаја овог наратива је покрет личног напретка. Оно што мету жртви \textit{redpill} покрета разликује је далеко веће осећање усамљености. Такво осећање, на жалост, сасвим је утемељено у стварности - број блиских пријатеља је све мањи.

Део који није утемељен у стварности је виђење комплетног непријатељства друштва према појединцу. Ово иде и до граница психолошког самоповређивања. На жалост, због одсуства видљивих ожиљака, појава је теже уочљива јавности, иако штета по појединца може да буде и већа у односу на физичко самоповређивање.

Чак ни Ендру Тејт, икона \textit{redpill} покрета, није нова појава. У књизи Познавање човека, Алфред Адлер је описао \textit{нападачки тип} личности. (прекуцај укратко из књиге)

Прича Ендру Тејта се своди на низ сличних изјава, чија је главна парадигма лична борба против свеприсутних неповољних околности. По његовом виђењу, мушкарац се рађа небитан и једини његов задатак је да се у животу оствари стичући поштовање од стране других мушкараца, али и жена. Једини начин за ово је победа у борби са другим мушкарцима, која се пре свега огледа у већој способности, снази и количини зарађеног новца. Такође, негира постојање депресије и анксиозности.

Депресију види као заверу чији циљ је уништавање мушкости. Сматра да је депресија осећање, и да као таква не може да буде стварна. Види је као израз себичности, пошто депресивна особа размишља о својим осећањима, која њега не занимају. Сматра да је за пораст депресије у свету криво слављење слабости. Анксиозност види као нормалну реакцију на неспособност појединца пред животним изазовима, наглашавајући да он не може да буде анксиозан јер је успешан у свему.

Негирањем важности људских осећања, као и сталним наглашавањем неопходности борбе против слабости, очигледно је да Ендру Тејт покушава да савлада своје лично осећање мање вредности. Наглашавање неопходности преузимања контроле осликава уплашеност пред непредвидивим околностима. Наглашавање важности посвећености послу и пословним успесима поред превазилажења личног осећања мање вредности осликава и нездрав дух времена у којем живимо.

Амерички психотерапеут Данијел Маклер (широј јавности познат по снимању документарних филомова о лечењу писхоза без употребе антипсихотика) увео је концепт \textit{фантазије о спасењу од стране родитеља} (parental rescue fantasy). По његовом виђењу, деца чије основне потребе у детињству нису задовољене од стране родитеља ову фантазију преносе у одрасло доба, очекујући да ће неко други да се брине за њих безусловно. Исто тако, задовољење ове фантазије је могуће пружањем помоћи, за коју се за узврат добија љубав и пажња.

Свеприсутност интернета учинила је да људи на све шири скуп питања одговор траже \textit{гуглањем}. Стога, није чудно очекивати да се одговори на животна питања на која се наилази на путу стицања зрелости такође траже на интернету, посебно од стране млађе популације која је склонија употреби технологије. Алгоритми друштвених мрежа су дизајнирани тако да нуде садржај сличан већ гледаном, чиме се додатно учвршћују виђења којима је корисник склон. Како се животне околности јесу нагло промениле, посматрање наратива различитог од погледа родитеља може да остави уверљив утисак.

Део публике је неоспорно незрео, што због својих година старости, или због одрастања под неповољним породичним и животним околностима. Други део је жртва преувеличавања и интернализовања захтева данашњице - бесконачног остварења, рада до изнемоглости. У оба случаја, штета је велика јер се у преломним животним тренуцима појединцу не нуди садржај који води стварном напретку и сазревању, већ садржај који води уверљивом привиду. Код млађих гледалаца, штета је далеко већа због мање развијене способности критичког размишљања.

За разлику од личног напретка, који појединца усмерава да бескорисне, али релативно безазлене навике (попут ...), \textit{redpill} уверења наносе директну штету...

\end{document}
