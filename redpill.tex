\documentclass[b5paper]{article}

\usepackage[serbianc]{babel}
\usepackage[b5paper, lmargin=0.1\paperwidth, rmargin=0.1\paperwidth, tmargin=0.1\paperheight, bmargin=0.1\paperheight]{geometry} %margins
\usepackage{float}
\usepackage{graphicx}
\usepackage{caption}
\usepackage{subcaption}
\usepackage{hyperref}

\edef\restoreparindent{\parindent=\the\parindent\relax}
\usepackage{parskip}
\restoreparindent

\newcommand{\websource}[1]{\caption*{Преузето са: \href{#1}{#1}}}

\begin{document}

\large

У претходном тексту, обрађен је феномен личног напретка. Тема овог текста је \textit{redpill} покрет, далеко радикланији и штетнији покрет који прати исту основну црту - превазилажење осећања мање вредности појединца. Покрет је свој назив добио по чувеној сцени из филма Матрикс. Главном лику Неу понуђен је избор између \textit{плаве пилуле} - живота у илузији и \textit{црвене пилуле} - живота у стварности.

За разлику од личног напретка, који појединца усмерава да бескорисне, али релативно безазлене навике (попут ...), \textit{redpill} уверења наносе директну штету...

\end{document}
