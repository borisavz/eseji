\documentclass[b5paper]{article}

\usepackage[serbianc]{babel}
\usepackage[b5paper, lmargin=0.1\paperwidth, rmargin=0.1\paperwidth, tmargin=0.1\paperheight, bmargin=0.1\paperheight]{geometry} %margins

\edef\restoreparindent{\parindent=\the\parindent\relax}
\usepackage{parskip}
\restoreparindent

\begin{document}

\large

Последњих година, у америчкој јавности се све више говори о резултатима истраживања која показују како је број блиских пријатеља становника САД у паду последње три деценије. 1990. године, 40 одсто мушкараца је изјавило да има 10 или више блиских пријатеља, док је 3 одсто изјавило да нема блиских пријатеља. 2020. године, број мушкараца који имају 10 или више блиских пријатеља је опао на 15 одсто, док је број мушкараца који немају блиских пријатеља порастао на 15 одсто\footnote{https://www.americansurveycenter.org/why-mens-social-circles-are-shrinking/}.

1990. године, 28 одсто жена је изјавило да има 10 или више блиских пријатеља, док је 2 одсто изјавило да нема блиских пријатеља. 2020. године, број жена које имају 10 или више блиских пријатеља је опао на 10 одсто, док је број жена које немају блиских пријатеља порастао на 10 одсто.

Други појам присутан у америчкој јавности, како стручној тако и широј је \textit{изгладнелост за додиром} (енгл. touch starvation, touch hunger). Под овим појмом, подразумева се стање неспокоја изазвано недостатком физичког додира. Овај појам не потиче из психологије или медицине, већ из колоквијалне употребе. Та чињеница не негира постојање појаве, иако је без истраживања немогуће одредити детаље о распрострањености.

Истраживања су показала везу између задовољавајућег друштвеног живота и бољег здравственог стања и дужег живота\footnote{https://www.ncbi.nlm.nih.gov/pmc/articles/PMC3150158/}. Такође, истраживања су показала како физички додир изазива лучење окситоцина, хормона (питај неког доктора)\footnote{https://elifesciences.org/articles/88215}. Да ли су ово закључци које можемо да прихватимо као такве, или постоји дубљи разлог иза закључака наведених истраживања?

Алфред Адлер (широј јавности познат по увођену појмова комплекса ниже и више вредности) је сматрао, угледајући се на закључке Чарлса Дарвина, да је због (релативне) физичке немоћи људима неопходан живот у заједници. Људи се рађају са осећањем немоћи које покушавају да савладају целог живота (провери у књизи како). Сматрао је да култура има кључни значај у опстанку човека и да је настала из неопходности живота у заједници.

Сматрам да оно што мучи Американце није само мањи број пријатеља или одсуство физичког додира. Прави проблем је све веће осећање отуђености, односно недостатка блискости. На жалост, индивидуализам и отуђеност су особине индустријализованих друштава. Одсуство заједништва за човека није природно, јер је управо живот у заједници обезбедио услове у којима људи данас живе.

\end{document}
