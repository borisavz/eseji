\documentclass[b5paper]{article}

\usepackage[serbianc]{babel}
\usepackage[b5paper, lmargin=0.1\paperwidth, rmargin=0.1\paperwidth, tmargin=0.1\paperheight, bmargin=0.1\paperheight]{geometry} %margins

\edef\restoreparindent{\parindent=\the\parindent\relax}
\usepackage{parskip}
\restoreparindent

\begin{document}

\large

Када причамо о Западу, најчешће је фокус на стању привреде и релативном заостатку Србије. Тема које је мало људи свесно је мрачна страна привредног развоја. У високоиндустријализованим државама света, у порасту је све већи осећај усамљености. Шта је ово проузроковало и постоји ли решење? Колики је удео проблема који је настао природно, а колики вештачки подстакнут? Иако је на оваква питања тешко дати целокупан одговор, усудио бих се да изнесем лични поглед.

Последњих година, у америчкој јавности се све више говори о резултатима истраживања која показују како је број блиских пријатеља становника САД у паду последње три деценије. 1990. године, 40 одсто мушкараца је изјавило да има 10 или више блиских пријатеља, док је 3 одсто изјавило да нема блиских пријатеља. 2020. године, број мушкараца који имају 10 или више блиских пријатеља је опао на 15 одсто, док је број мушкараца који немају блиских пријатеља порастао на 15 одсто\footnote{https://www.americansurveycenter.org/why-mens-social-circles-are-shrinking/}.

1990. године, 28 одсто жена је изјавило да има 10 или више блиских пријатеља, док је 2 одсто изјавило да нема блиских пријатеља. 2020. године, број жена које имају 10 или више блиских пријатеља је опао на 10 одсто, док је број жена које немају блиских пријатеља порастао на 10 одсто.

У тренутку писања овог текста, није било доступних упоредивих истраживања на нашим просторима (Србија и бивша Југославија). По једином доступном истраживању над адолесцентима узраста од 13 до 18 година, 37.6 одсто адолесцената је изјавило да има више блиских пријатеља , 53.8 одсто да има неколико блиских пријатеља, 6.9 одсто да има једног блиског пријатеља, док 1.7 одсто нема ниједног блиског пријатеља\footnote{https://www.dps.org.rs/wp-content/uploads/2023/08/Kongres-psihologa-2023-Knjiga-rezimea-2023-08-26.pdf}. Важно је нагласити да због ограничене старосне групе, истаживања нису упоредива.

Други појам присутан у америчкој јавности, како стручној тако и широј је \textit{недостатак додира} (енгл. touch starvation, touch hunger). Под овим појмом, подразумева се стање неспокоја изазвано недостатком физичког додира. Овај појам не потиче из психологије или медицине, већ из колоквијалне употребе. Та чињеница не негира постојање појаве, иако је без истраживања немогуће одредити детаље о распрострањености.

Овај појам је присутан и у домаћој јавности, како у новинама, тако и у блог објавама стручних лица (психијатара и психотерапеута). У жижу јавности појам је дошао са епидемијом корона вируса и закључавањем.

Истраживања су показала везу између задовољавајућег друштвеног живота и бољег здравственог стања и дужег живота\footnote{https://www.ncbi.nlm.nih.gov/pmc/articles/PMC3150158/}. Такође, истраживања су показала како физички додир изазива лучење окситоцина, хормона (питај неког доктора)\footnote{https://elifesciences.org/articles/88215}. Да ли је довољно да резултате истраживања прихватимо као такве, или је потребно да у њима потражимо дубљу поуку?

Алфред Адлер (широј јавности познат по увођену појмова комплекса ниже и више вредности) је сматрао, угледајући се на закључке Чарлса Дарвина, да је због (релативне) физичке немоћи људима неопходан живот у заједници. Људи се рађају са осећањем немоћи које покушавају да савладају целог живота (провери у књизи како). Сматрао је да култура има кључни значај у опстанку човека и да је настала из неопходности живота у заједници.

Суштински проблем није само мањи број пријатеља или одсуство физичког додира. Прави проблем је све веће осећање отуђености, односно недостатка блискости. На жалост, индивидуализам и отуђеност су особине индустријализованих друштава. Одсуство заједништва за човека није природно, јер је управо живот у заједници обезбедио услове у којима људи данас живе.

Тренутна опасност по човечанство је да долази до сукоба дугорочног интереса човечанства и краткорочног интереса потрошачког друштва. Дугорочно, човечанство је одрживо искључиво уколико појединци могу да напредују кроз стваралаштво и сарадњу са друштвом. Краткорочно, произвођачи бројних бескорисних производа не би остварили ни тренутак пажње јавности уколико би фокус био на стваралаштву, уместо потрошњи.

Суштина стваралштва је остављање дела које превазилази живот ставароца, као и времена у којем живи. Због тога, стваралаштво је нераскидиво са осећањем заједништва. Насупрот томе, суштина потрошачког друштва је занемаривање било кога осим појединца и његових жеља. Иако на први поглед делује да појединац може да ужива у бесконачној себичности која му се препоручује, резултат је сасвим супротан.

Осећај заједништва и стваралштво никада неће бити вредности које ће потрошачко друштво подстицати. Човек којем су на првом месту породица, пријатељи и стваралачки рад никада неће трошити новац на непотребне производе којима су тржни центри преплављени. Али, оваквом системском променом приоритета човечанство залази у ћорсокак. На личном плану, резултат ће бити отуђени и несрећни појединци. На глобалном плану, резултат ће бити нефункционално друштво које није у стању да створи напредак.

Никако не смемо потцењивати ни дигиталне индустрије које директно злоупотребљавају усамљеност. Вероватно најјачу индустрију "усамљености" чине порнографски сајтови. Иако порнографија није нова појава, бесконачна доступност путем интернета, као и зависност од погрнографије јесу. Иако тема отворене дебате, сматрам да је утицај порнографије штетан.

Као узрок опадања осећаја заједништва, видим комбинацију неповољних животних околности и неадекватног система вредности. За формирање пријатељстава, пре свега је потребно континуирано провођење времена у колективу. Због школовања и посла, људи су све чешће приморани на селидбу. Рад од куће, иако је неоспорно донео предности, значајно онемогућава друштвени живот. Упарено са општим неразумевањем сопствених потреба, за које поред индустрије морамо да окривимо и школски систем, резултат је нарастајуће осећање усамљености.

Са нарастањем изазова свакодневног живота, неминовно ће расти и притисак који трпе појединци. Додатно отежавајућа околност по стање појединаца биће култура која не задовољава основне човекове потребе за заједништвом и разумевањем. Поучени дешавањима данашње Америке, можемо поставити животне приоритете. Наравно, решавање овако суштинског проблема у неповољном окружењу неће бити једноставан задатак.

\end{document}
