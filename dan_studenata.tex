\documentclass[b5paper]{article}

\usepackage[serbianc]{babel}
\usepackage[b5paper, lmargin=0.1\paperwidth, rmargin=0.1\paperwidth, tmargin=0.1\paperheight, bmargin=0.1\paperheight]{geometry} %margins

\edef\restoreparindent{\parindent=\the\parindent\relax}
\usepackage{parskip}
\restoreparindent

\begin{document}

\large

За дан студената, одлучио сам се на писање критике погледа већине наших студената на сврху студија, као и на концепт студентског живота. Иако је тема контроверзна, сматрам да је јако важан део шире теме изградње система вредности, као и улоге школства у том задатку. Иако комунизам није нарочито популаран међу данашњом студентском популацијом, и упркос што је на студентским окупљањима и журкама често могуће видети српске националне симболе као и чути родољубиве песме, стил живота наших студената је директно преузет из доба комунизма.

Пошто сам био студент информатике и живео у студентском дому и којем су већина станара били студенти техничких наука, мој поглед је пристрасан. Замолио бих читаоце да се уздрже од прераних осуда. Иако уистину јесам мање упознат са дешавањима ван свог факултета, примећујем да ситуација ипак није драстично боља. Истина може да боли и верујем да ће део студената бити увређен или збуњен. \textit{Ко је овај деда и зашто смара?} На писање критике сам се одлучио јер сматрам да осим личне судбине, погрешно усмеравање од ране младости негативно утиче и на будућност Србије.

Као и већина студената, морао сам да из малог места (Гроцке) дођем у велики град (Нови Сад) на студије. Убрзо сам научио да студентски дом није старачки дом, те да би требало да се навикнем на свакодневну буку и журке. Како сам по природи скептик, кроз студије сам се водио изреком да не треба слушати савете других студената, а поготово не оних вечитих. Убрзо сам сазнао да су \textit{најгласнији} пренели превише испита да би остали у дому. Након опаски да ми праћење предавања није потребно, решио сам да потражим себи боље друштво.

Ипак, морао сам да се запитам који је узрок два фундаментално другачија поглед на студије. Сматрам да одговор лежи у одређеним социјалним и политичким аспектима живота на нашем поднебљу. Осим што долазе из малих места и села, добар део студената је из радничких породица. У овоме нема ништа лоше, већ напротив представља велики потенцијал за развој наше земље. Иако трошкови становања у граду (за студенте који не живе у дому), као и самофинансирања нису мали у односу на просечна примања, усудио бих се да кажем да је студирање ипак далеко лакше приуштити у односу на Америку. По мојем личном мишљењу, доступност школства је кључна за спречавање неконтролисаног економског раслојавања. На жалост, наши студенти ово не разумеју и због тога своје време проведено на студијама не користе ефикасно.

Проблем је што када дођу у нову средину, бруцоши не разумеју сасвим шта нова средина представља. Живот у граду пре свега виде као прилику за провод, и сам долазак у град виде као довољан успех. Како се од бруцоша очекује да на факултету наставе стицање нових знања из жељене области (иако се у пракси увек очекује одређени ниво предзнања чак и из непознатих области), сулудо је кривити искључиво студенте за њихово (незрело) понашање. Иако, уистину, бруцоши више нису код учитељице, да ли би улога универзитетских професора требало да буде делом и васпитна? Иако сам се лично определио за рад у индустрији, незамисливо ми је да професор треба да остане нем на дух који негативно утиче на резултате испита.

Са порастом животног стандарда у комунистичкој Југославији, крајем седамдесетих и почетком осамдесетих година 20. века, настао је концепт студентског живота. Вероватно први пут у историји, школовање је постало асоцијација за забаву, а не рад. Иако Југославије одавно нема, концепт студентског живота се и даље пажљиво негује. Одмах по упису на студије, бруцоши су бомбардовани са свих страна бруцошијадама, електријадама и позивима на учлањење у безброј студентских организација.

Овиме се код младих ствара једна погрешна концепција, како школовања, тако и живота. Додатно, овиме се нарушава углед школства. Али, највећа је ипак штета која се наноси појединцима који подлежу овом штетном утицају. ИТ индустрија је окрутно и веома такмичарски оријентисано окружење. Какве су шансе за опстанак уколико одбацимо учење лекција које се тешко уче по завршетку школовања, а које су неопходне?

Пораст животног стандарда је увек био праћен настанком потрошачког друштва. Настанак потрошачког друштва на нашим просторима је био шездесетих година 20. века. И даље, у нашем народу потрошачки идеали из периода комунизма попут одласка на море живе. Границе ове опсесије су готово комичне! Иако су данашња времена далеко од благостања, потрошачко друштво пружа бројне погодности онима који желе да буду просечни. Ово негативно утиче на вољу за стваралаштвом. Готово је немогуће објаснити некоме зашто би требало учити, или радити, ако је животни стандард довољан да обезбеди куповину увозне одеће или одлазак на море.

Лично никада нисам разумео опсесију студената за учлањење у студентске организације. Претпостављам да се ради о сплету неколико здравих тежњи: жеље за друштвеним животом, желе за хобијима као и жеље за повезивањем са људима из струке. Такође, студенти су често доведени у заблуду како је навођење чланства или учествовања у огранизационим активностима корисно при запошљавању. Истина је да се послодавци не обазиру и то не сматрају релевантним искуством. Моја критика се односи на студентске организације намењене студентима електротехнике и рачунарства, првенствено. Као студент, радо сам са друштвом са психологије ишао на радионице које огранизују њихове организације, и које су поучне и примерене.

Студенти информатике су занимљиви за анализу управо из разлога што је често видно да нису на студијама због свесне воље за бављењем ИТ-јем. Наравно, није никакав проблем уколико неко не жели да се бави одређеним занимањем. Проблем је што повећање уписних квота мора да прати и повећање броја дипломаца. Константно обарање критеријума наноси штету оним студентима који заиста јесу талентовани и заинтересовани. Давање прилике свакоме да се школује је здрав идеал којем тежи свака држава која има амбиције за остваривање одрживог напретка. Погрешно је омогућити да свако може да заврши школу. Другим речима, школству је потребна \textit{једнакост прилике}, а не \textit{једнакост остварења}.

Када сам био дете, још увек се могао чути одјек времена када је програмирање било авангарда. Та времена су, на жалост, одавно прошла. Сада је одјек значајно угушен буком маркетинга. Данас би свако желео прогамерске паре, али нико није свестан колики бол у души носе програмери. Због тога, са жаљењем посматрам све мању жељу за стваралаштвом код студената. Иако рачунарске науке нису једина погођена област, проблем је нешто израженији јер тренутне економске прилике подстичу људе на упис студија из ове области.

Тако већ годинама гледам како студенти, по завршетку живота из бајке, често налазе послове који по захтевности не одговарају универзитетском образовању. Разлог овоме нису искључиво тренутне економске прилике. Студенти који од почетка студија нису разумели смисао студија, нити су имали жељу за нешаблонским размишљањем, ни по завршетку студија не виде разлику између \textit{озбиљног} и \textit{неозбиљног} посла. Чак су ми се и професори за време студија жалили како по савладавању основих лекција, студенти губе интересовање. Увек се растужим када видим колики број дипломаца носи звања инжењера рачунарства, а да притом не разуме суштинске концепте сакривене испод технологија које свакодневно користе. Јер ако то није посао инжењера, шта је?

Времена која очекују човечанство су кризна. Али криза неће бити само материјална, него пре свега духовна! Једине вештине које ће моћи да нас спасу су знање и способност за сагледавање шире слике! На жалост, нити животне околности, нити школство, не подстичу формирање наведених вештина! Док нам стварност не буде ударила шамар, неопходно је подићи свест о изазовима савременог доба. Са развојем вештачке интелигенције, део шаблонских занимања ће бити елиминисан. Али људску аутентичност није тако лако заменити и она ће морати да остане на цени!

Сматрам да би у здравом друштву главну улогу у формирању система вредности требало да имају родитељи. Али никако не смемо занемарити ни улога школства. Ипак, времена се мењају и не можемо очекивати од родитеља да се баве свиме. Неопходно је да факултети престану да себи нарушавају углед промовисањем пијанки и да као алтернативу понуде образовне ваннаставне активности. Омладина представља будућност сваке државе, и узалудно је уздати се у развој науке и технологије уколико је осим самог квалитета рада неадекватан поглед на улогу и важност рада.

Не сматрам да је решење увести правила слична америчким универзитетима која експлицитно забрањују сарадњу између студената на изради студентских пројеката. Системе вредности засноване на такмичењу и непријатељству сматрам дубоко погрешним. Свако велико дело ствралаштва у науци и техници створено је сарадњом. Наравно да решење није ни толерисање преписивања и варања на испитима. Довољно би било поставити критеријуме који одговарају универзитетском образовању. Самим тиме би и стил живота као и ниво посвећености и љубави према науци морао да прати ниво који се очекује од студената.

Без националне интелигенције не можемо очекивати ни националну обнову. На жалост, студентске журке, махање заставом и провлачење на испитима не могу да произведу националну интелигенцију. Историја српског народа је била бурна и хватање корака са светом у неповољним околностима никад није био једноставан задатак. У књизи \textit{Чујте, Срби!}, Арчибалд Рајс је критиковао српске студенте у иностранству. Сматрао је да је знање остајало површно, књишко и \textbf{да се није стапало са духом}. Иако данас имамо своје универзитете, критика остаје релевантна. Ипак, здраве темеље још увек имамо. На нама је да одлучимо - хоћемо ли да жмуримо, или ћемо да се усудимо и да погледамо у провалију пред којом стојимо?

\end{document}


